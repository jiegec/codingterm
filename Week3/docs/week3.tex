% Created 2018-09-14 Fri 14:24
% Intended LaTeX compiler: pdflatex
\documentclass[11pt]{article}
\usepackage[utf8]{inputenc}
\usepackage[T1]{fontenc}
\usepackage{graphicx}
\usepackage{grffile}
\usepackage{longtable}
\usepackage{wrapfig}
\usepackage{rotating}
\usepackage[normalem]{ulem}
\usepackage{amsmath}
\usepackage{textcomp}
\usepackage{amssymb}
\usepackage{capt-of}
\usepackage{hyperref}
\usepackage[newfloat]{minted}
\usepackage{xeCJK}
\setCJKmainfont{Songti SC}
\usepackage{latexsym}
\usepackage[mathscr]{eucal}
\usepackage[section]{placeins}
\usepackage{float}
\usepackage{svg}
\author{计72 陈嘉杰}
\date{\today}
\title{Python 大作业 新闻抓取大作业}
\hypersetup{
 pdfauthor={计72 陈嘉杰},
 pdftitle={Python 大作业 新闻抓取大作业},
 pdfkeywords={},
 pdfsubject={},
 pdfcreator={Emacs 26.1 (Org mode 9.1.14)}, 
 pdflang={English}}
\begin{document}

\maketitle
\tableofcontents

\section{题目说明}
\label{sec:org458c8a6}
通过 Python 实现新闻网站的抓取,并通过 Django 框架实现新闻的在线搜索和相关新闻推荐功能。

\section{功能实现}
\label{sec:orga192441}
\subsection{新闻主页}
\label{sec:orgcb97a75}
显示一个类似谷歌搜索主页的页面,提供搜索、Feeling Lucky和所有新闻显示三个按钮。
\subsection{新闻抓取}
\label{sec:orga6fcd9c}
\subsubsection{在线接口}
\label{sec:org0b722d0}
在 \texttt{/news/scrape} 下提供一个表单,用户可以提交新闻页面,请求新闻的抓取。如果抓取成功,则调转到这个新闻的详情页。
\subsubsection{批量抓取}
\label{sec:org0573c6a}
编写了 \texttt{scraper.py} ,在腾讯新闻网上,通过三种途径获得新闻的列表,并且对每条新闻进行抓取,保存到数据库中。
\subsubsection{新闻处理}
\label{sec:org366fbcc}
下载新闻内容,解析后保存到数据库中,并建立倒排索引。
\subsection{列出所有新闻}
\label{sec:org1e75b93}
实现了支持分页功能的全部新闻展示功能,显示每条新闻的源地址、标题、发布时间和摘要。点击标题即可进入该新闻的详情页。数据库中目前已有一万多条新闻的数据。提供了获取新闻时间的显示,一般在 10\(^{\text{-3}}\)s 的量级。
\subsection{新闻详情页}
\label{sec:org66791b1}
通过编写 \texttt{CSS} ,将抓取到的新闻美观地显示,并且在页面底部通过延迟加载的方式,获取相关新闻推荐。
\subsection{新闻搜索}
\label{sec:org8fc7be3}
输入搜索关键字,后端进行分词后,显示带有这些关键字的新闻。同时支持时间范围的搜索,通过发布时间进行过滤。提供了搜索耗时的显示,一般在 10\(^{\text{-2}}\)s 的量级,体验良好。

\section{程序运行}
\label{sec:org141c57a}
\begin{enumerate}
\item 操作系统: macOS
\item 解释器: CPython 3.7.0
\item 依赖:Django 2.1.1 BeautifulSoup4 4.6.3 requests 2.19.1
\end{enumerate}


\section{实现思路}
\label{sec:org3b1e4b7}
通过 \texttt{Django} 的 \texttt{ORM} 进行数据的储存,通过 \texttt{BeautifulSoup4} 和 \texttt{Requests} 进行网站的抓取和解析,通过 \texttt{Jinja} 进行静态网页的渲染,通过 \texttt{TF-IDF} 和 \texttt{Jaccard Index} 进行新闻的搜索和相关新闻的推荐。

\section{实现细节}
\label{sec:org74b12eb}
\subsection{新闻抓取}
\label{sec:orgdd003ea}
\subsubsection{批量抓取}
\label{sec:orge252821}
通过对腾讯新闻网页的分析,找到它所使用的几处 \texttt{XHR} 请求,通过直接请求这些页面,可以获得一个较为格式化的新闻列表。并且通过腾讯新闻的指定日期滚动新闻功能,可以批量抓取指定日期的新闻。为了防止爬虫访问受限制,代码采用了自定义 \texttt{HTTP Header} 的方式,并且采用了随机 \texttt{User-Agent} 轮换的方法和指数退却(Exponential Backoff)的超时策略。
\subsubsection{新闻处理}
\label{sec:orge3269bc}
首先,将网页内容下载下来,使用 \texttt{BeautifulSoup4} 进行解析。支持四类已知的新闻页面,它们的主要区别在于,正文、发布时间等采用的 HTML 标签不同,也有的页面是将内容保存在 Javascript 中,代码中都进行了判断和提取,将无关的一些标签去除后,获取完整新闻文本,对此进行分词,并生成摘要。将连接、提取到的标题、正文、发布时间和使用 \texttt{jieba} 分词得到的单词存入数据库,建立倒排索引。倒排索引采用 \texttt{ManyToManyField} ,即新闻和单词的多对多关系,方便双向的查找。
\subsection{列出所有新闻}
\label{sec:orgb3a681f}
实现了分页功能,允许指定每页有多少文章,和该页从哪篇文章开始,并在页面底部提供了上一页、下一页的连接。显示每条新闻的源地址、标题、发布时间和摘要,利用 \texttt{CSS} 模仿谷歌的搜索页面,点击标题即可进入该新闻的详情页,点击源连接即可进入新闻的源地址。
\subsection{新闻详情页}
\label{sec:org47e69e5}
通过 \texttt{CSS} ,将抓取到的正文进行正常显示,同时提供重新抓取功能,即可以要求后端对该新闻进行重新抓取,方便代码在更新后重新抓取指定页面。页面底部通过 \texttt{iframe} 获取当前新闻的相似新闻,使得在后台进行推荐算法的计算时,用户可以查看新闻全文。
\subsection{相似新闻}
\label{sec:org5e6d0ce}
实现了新闻推荐的在线算法。首先,根据当前新闻的正文,通过 \texttt{BeautifulSoup4} 过滤掉一些标签后获取文本,通过 \texttt{jieba.analyse.extract\_tags} 获取 \texttt{TF-IDF} 指数高的词语,查询数据库获取到含有这些词语的新闻,对每条新闻,计算 \texttt{Jaccard Index} 作为这条新闻的权值,最后显示权值最高的三条新闻作为推荐结果。在优化推荐算法搜索时间和搜索结果上,进行了诸多尝试,包括减少数据库查询次数、将一些运算移至数据库中进行等等。
\subsection{新闻搜索}
\label{sec:org51109ae}
对输入进行分词,对每个关键词,通过倒排索引,搜索到相关的新闻,并且根据关键词出现的新闻次数和关键词出现在该新闻的次数求出 \texttt{TF-IDF} 指数,作为权值对新闻进行排序,其中如果某个关键词出现的新闻次数过多,对应的权值会大幅减少,类似于停用词的处理,然后按照分页的请求显示部分结果,并将搜索的关键词进行替换,从而实现高亮显示。经过多次尝试,采用对数计算的 \texttt{TF-IDF} 指数效果较好,并对 \texttt{IDF} 的停用词采用了 \texttt{0.5} 的阈值。
\end{document}